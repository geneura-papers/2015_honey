\documentclass[a4paper]{llncs}

\usepackage{times,verbatim} % Please do not comment this
\input{psfig.sty}

\begin{document}

\pagestyle{empty}

\mainmatter

\title{Cybersecurity in TIC-Scientist Network Infrastructures\\by Honeypots:\\Catching Cyber Threat Passively}

\titlerunning{Cybersecurity in TIC-Scientist Network Infrastructures by Honeypots}

\author{Juan Luis Martin Acal\inst{1} \and Pedro A. Castillo Valdivieso\inst{1}
\and Gustavo Romero López}

\authorrunning{Juan Luis Martin Acal}

\institute{Springer-Verlag, Computer Science Editorial III,
Postfach 10 52 80,\\
69042 Heidelberg, Germany\\
\email{jlmacal@correo.ugr.es}\\
\email{\{pcv, gustavo\}@ugr.es}\\
\texttt{http://www.springer.de/comp/lncs/index.html}
}

\maketitle

\begin{abstract}
There is a balance between security worries and right to privacy. Universities have a high risk of attack as a source of valuable information. Private and scientific information have a enormous value for an attacker but end user is worry about his privacy too. For this reason passive detection methods in cybersecurity like honeypots are the cornerstone in the defence plan. We expose the practical case of the University of Granada in the application of honeypot for the detection and study of intrusions.
\dots
\end{abstract}


\section{Introduction}

From earliest days, the networks have been experiencing an increasing number of attacks. Nowadays, the number of attacks increases continuously and scientist networks are a stage very interesting. There is a strong demand of security in the network and the services which are listening. In the other hand, the end users demand privacy in his network traffic. In this scene the honeypots have an important role in the detection and protection against cyber attacks.

\subsection{Cyber-Space and Cyber-Threats}
\label{sect:Scientist Networks}

\subsection{Scientist Networks}
\label{sect:Scientist Networks}

\subsection{Privacy and Passive Sensors}
\label{sect:Privacy and Passive Sensors}

Springer-Verlag now provides the full-text version of
the LNCS and LNAI proceedings online. Therefore please
submit to the {\it volume editors} (and not to Springer-Verlag),
together
with your own single-sided printout of the final version of your
contribution (which cannot be modified at a later stage), your source
(input) files, e.g. TEX files for the text
and PS or EPS files for figures, the final DVI file (for papers prepared
using \LaTeX\ or \TeX), the final
PS file\footnote{When generating the PS file please avoid using the
	option ``reverse order".}, and, if possible, a PDF file of the final
version of your contribution.
If you have prepared your paper using a text processing system other
than \LaTeX\ or \TeX, please also submit RTF files.
Make sure that the text is {\em identical} in all cases.


\section{Deploy of a S.I.E.M Based in Honeypots}

\section{Weaknesses and Strengths of Honeypots}

\section{Honeypots, Elements in Hybrid Machine Learning S.I.E.M}

\section{Conclusions and Future Works}

\end{document}
